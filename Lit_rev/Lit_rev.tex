\documentclass{article}

\usepackage[utf8]{inputenc} %Only required if not using XeLatex
\usepackage[T1]{fontenc}
\usepackage[final]{microtype}
\usepackage{bm} %Bold font for math equations

\usepackage{verbatim} %For word count
\newcommand{\detailtexcount}[1]{\immediate\write18{texcount -merge -sum -nobib  #1.tex output.bbl > #1.wcdetail}\verbatiminput{#1.wcdetail}}

\usepackage{helvet} %Set font type (helvetica)
\renewcommand{\familydefault}{phv} %Set helvetica as default font

\usepackage[a4paper, margin=25mm]{geometry} %Setting document size and margin width

\usepackage[url=false, doi=false, isbn=false, bibstyle=ieee, citestyle=numeric-comp]{biblatex} %Set bibliography style
\addbibresource{Lit_rev.bib} %Imports .bib file

\usepackage{graphicx} %Required for graphics and figures
\graphicspath{ {./Fig/} }
\usepackage[export]{adjustbox} %Used for adjusting image box
\usepackage{caption} %For caption customizations
\usepackage{subcaption}

%TC:ignore
\title{\textbf{Uncertainty Quantification of Plasma Filament in the Tokamak Scrape-off Layer}}
\author{Jasper Ng}
\date{May 2025}
%TC:endignore

\begin{document}

\maketitle

\section*{Lay Summary}
Fusion is the process of fusing two atoms together, releasing energy in the process. Due to the conditions required for fusion to occur, most contents within will change into a state called plasma. Plasma is a state where gaseous atoms are ionised and electrons are stripped from the atoms. In a magnetic fusion device that is usually a doughnut-shaped torus, fusion occurs within the plasma confined within the device using strong magnetic fields. Under the influence of strong magnetic fields and instabilities that cause perturbations in the density of the plasma, blobs of plasma called plasma filaments or blobs would be ejected from the edge and hit the sidewall of the vessel. Hence, increasing amounts of research in recent years has focused on investigating the dynamics of how this blob evolves as it travels in the vessel. To describe the dynamics of the evolution of a blob, mathematical models were developed and computational simulations were performed and compared with experimental data to test the models.

Mathematical models are a framework of equations that are used to describe the properties of the subject of interest that varies in time. In the context of plasma filaments, the model describes the evolution of the blob and could be used to find properties such as density, size and velocity of the blob. Models require inputs to produce output, and each input individually and through the interaction with other inputs will have a varying impact on the output. The impact of each parameter would be ranked to review how much impact it would have on the output result when changing the particular input result.

Sensitivity analysis is a method of analysis to determine the impact of the inputs have on the model output. Each input would be changed and fed into the model to compute the output, and this process would be repeated a large number of times to determine the impact for each input parameter. This would require a lot of resources or time in terms of computation to evaluate if the full model were used, making it computationally expensive. Therefore, a faster, lighter surrogate model that could compute similarly to the full model that requires less resource to compute would be used to perform the task.

In this work, models that output the properties of plasma filaments would be investigated to determine the impact each input has on the model output. To reduce the computational cost required for the large number of model evaluations, a surrogate model would be used.    

\section*{Abstract}
Plasma filaments or blobs evolution dynamics have been the focus of recent studies for magnetic fusion devices, and an increasing amount of interest have been given to simulation of the blobs. A common model used in 2D simulations is the sheath limited model and requires input of parameters such as density (n), vorticity ($\Omega$) and plasma length scales ($\delta$). In this work, the importance of each model parameter and interactions between the model parameters were determined and ranked through the application of uncertainty quantification using the Sobol method and a surrogate model such as Gaussian Processes (GP).

\section*{Literature Review}
\subsection*{Overview}
Plasma filaments or blobs are coherent plasma structures that form from perturbations in the Scrape-off Layer (SOL) that transport heat and material across the magnetic field lines \cite{dippolito_convective_2011, hoare_dynamics_2019}. It is useful to perform simulations on the evolution of plasma filaments as it could help to review particle and heat transport at the SOL, as well as the impact on the durability and lifetime of plasma-facing materials and vessel walls \cite{carralero_experimental_2015, krasheninnikov_recent_2008}. 

Uncertainty quantification is a framework that could model the uncertainties of the governing model parameters and the effect it has on the overall system \cite{sudret_surrogate_2017}. Using surrogate models such as Gaussian processes, sensitivity analysis could be performed on the parameters of the governing equations relating to the properties of the plasma filaments.

Through the application of uncertainty quantification, the parameters of the equations that govern plasma filament property could be analysed whilst avoiding being computationally expensive. 

\subsection*{Plasma Filament}
Plasma filaments are a common feature among magnetic fusion devices in different operating modes, appearing in devices such as tokamaks and stellarators and in operating modes including L-mode, H-mode and edge localised modes (ELMs) \cite{ben_ayed_inter-elm_2009, killer_plasma_2020, boedo_transport_2003}. The importance of this feature was due to the discovery of plasma recycling within the main chamber of the reactor, and instead of the fusion plasma flowing to the divertor it was flowing into the chamber walls \cite{krasheninnikov_recent_2008,dippolito_convective_2011}.

The mechanism for blob formation was proposed by Krashennikov \textit{et al.} \cite{krasheninnikov_recent_2008, krasheninnikov_scrape_2001}. At first, the plasma will be separated and formed at the outmost layer of the SOL due to turbulent processes. Plasma polarization of the separated plasma will occur due to particle drift effects within the vessel, such as the curvature drift and the $\bm{\nabla B}$ drift. The vertical charge separation then induces an $\bm{E}$ field in the perpendicular direction to the toroidal direction, leading to an $\bm{E}\times\bm{B}$ force on the plasma radially outward. An alternative mechanism of the current source was proposed by Omotani \textit{et al.} that a current source is due to the diamagnetic current drift in the plasma instead of the particle drifts. To maintain quasineutrality in the plasma, the circuit is closed through polarisation current or parallel current configuration.

As reviewed by \cite{dippolito_convective_2011} large amounts of 2D and 3D simulations have been done to understand blob dynamics with different closure schemes for the 2D model that explains the motion and properties of the blob. The most common closure scheme has been the sheath limited scheme, assuming the filament is in the far SOL and parallel current of the blob would be limited by sheath resistivity. Other examples of closure schemes focus on the effects of enhanced polarization at X-points and magnetic field line bending due to high $\beta$ plasmas\cite{krasheninnikov_recent_2008}. Work was done to expand the blob models to include other physical phenomena such as drift wave instabilities \cite{angus_effect_2012} and the cause of "Boltzman-like" potential due to blob spinning \cite{angus_effects_2012}, enabling the 3D effects of blob dynamics to be implemented into the evolution of the 2D blob simulations.

2D and 3D simulations were subsequently compared to review the accuracy of the 2D simulations to the 3D simulations. It was concluded in \cite{angus_3d_2012} that the 2D simulation with sheath limited closure is only valid on time scales that were short compared to the time for the drift wave development. More agreeable results between 2D and 3D simulations were found in \cite{easy_three_2014} for sheath dissipation closure and in some cases for vorticity advection closure, but both articles concluded that 2D simulations could not replicate 3D simulation dynamics when drift wave turbulence and blob spinning effects become dominant.

Since this project will be using the BOUT++ framework \cite{dudson_bout_2009} for blob simulation, the governing equations and relevant parameters for blob dynamics are referenced from the work by \cite{omotani_effects_2015}. The mathematical model used in this work considered that the filament velocity would be limited by inertial and sheath currents, with the assumption that parallel resistivity would be insignificant for parallel currents to reach the sheath. The variables that describe blob properties include density (n), vorticity ($\Omega$), plasma length scales (in the x and z directions, $\delta_x$ and $\delta_z$) and maximum filament velocity ($V_f$), with $V_f$ being the output variable scaling with plasma filament amplitude compared to the background (A), and $\delta_x$ or $\delta_z$, depending on the regime.  

\subsection*{Uncertainty Quantification}
Uncertainty quantification is the process that quantifies the uncertainties in the output by evaluating the uncertainties of the input of a mathematical model, that "deals with assessing, characterising and managing uncertainty in computer models and simulations" \cite{wu_chapter_2024,li_gaussian_2025}. The uncertainty assessed can be used for functions such as optimisation, sensitivity analysis and parameter estimations for the model. In the context of this project, uncertainty quantification will be used to perform sensitivity analysis on the input parameters for the model under the BOUT++ framework to assess the importance and influence of each parameter on the dynamics of blob simulation.

Sensitivity analysis (SA) is a process that evaluates the contribution of a model parameter to the response of the model. There are two main types of SAs including local sensitivity analysis (LSA) and global sensitivity analysis (GSA). The difference between the two is due to LSA evaluating parameter contributions individually, whereas GSA varies all parameters simultaneously, allowing evaluation of contributions from individual parameters and parameter interactions \cite{tosin_tutorial_2020, zhang_sobol_2015}. In particular for modelling plasma filament dynamic, the model is non-linear. Hence, to perform SA on the model input parameters, the contributions of individual parameter and interactions between parameters would have to be considered. Therefore, GSA methods would be preferred. 

Sobol method is a common GSA method for performing SA on the model parameters and has been applied to models in other fields, including biomechanics \cite{wirthl_global_2023} and pharmacology \cite{zhang_sobol_2015}. It is a variance-based method and divides the output variance into the contributing parameters of the model. As stated in \cite{wirthl_global_2023} and \cite{zhang_sobol_2015}, a sizeable sample is required for a reliable Sobol SA, and even smaller samples requiring ~1000 evaluations. This would be computationally expensive for high-fidelity models such as BOUT++, hence a less computationally expensive surrogate model would be desirable.

Common surrogate models include functions such as polynomial chaos expansion, gaussian processes (GP) and low-rank tensor approximations that emulate the high-fidelity model with low computational cost \cite{sudret_surrogate_2017}. GP are chosen for this project as smaller prediction error was found at lower dimensions \cite{gammel_gaussian_2024}, and that the method has already been applied in a similar topic of the same field \cite{hornsby_gaussian_2024}. 

As stated in \cite{wirthl_global_2023}, GP must be trained before it could be used as a surrogate model. The training process includes using generated sample inputs and conditioned values computed by the full model to form and train the GP. Kernels are functions that give the correlation between pairs of variables and define the shape of the GP. Different kernels could be combined to produce a GP that best emulate the model \cite{gortler_visual_2019, duvenaud_automatic_2014}.

\subsection*{Conclusion}
In conclusion, plasma filaments are an important and active area of study due to the damage that could be inflicted on vessel walls, and 2D simulations of plasma filaments could help to investigate the evolution of the blobs. In order to determine the importance of different parameters of the model used in the 2D simulation, uncertainty quantification through the application of Sobol method and a surrogate model such as Gaussian processes would be utilised in this work to rank the impact of the model parameters individually and in relation to other parameters whilst keeping the computational cost low.

\nocite{*}
\printbibliography[title={References}]

%TC:ignore
\detailtexcount{Lit_rev}
%TC:endignore

\end{document}

