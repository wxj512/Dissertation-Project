\documentclass{article}

\usepackage[utf8]{inputenc} %Only required if not using XeLatex
\usepackage[T1]{fontenc}
\usepackage[final]{microtype}
\usepackage{bm} %Bold font for math equations

\usepackage{verbatim} %For word count
\newcommand{\detailtexcount}[1]{\immediate\write18{texcount -merge -sum -nobib  #1.tex output.bbl > #1.wcdetail}\verbatiminput{#1.wcdetail}}

\usepackage{helvet} %Set font type (helvetica)
\renewcommand{\familydefault}{phv} %Set helvetica as default font

\usepackage[a4paper, margin=25mm]{geometry} %Setting document size and margin width

\usepackage{enumitem} %Custom list
\newlist{arrowlist}{itemize}{1} %Define custom list name
\setlist[arrowlist]{label=$\rightarrow$} %Define arrow for custom list label

\usepackage[url=false, doi=false, isbn=false, bibstyle=ieee, citestyle=numeric-comp]{biblatex} %Set bibliography style
\addbibresource{Lit_rev.bib} %Imports .bib file

\usepackage{graphicx} %Required for graphics and figures
\graphicspath{ {./Fig/} }
\usepackage[export]{adjustbox} %Used for adjusting image box
\usepackage{caption} %For caption customizations
\usepackage{subcaption}

%TC:ignore
\title{\textbf{Literature Review \\and \\Lay Summary}}
\author{Jasper Ng}
\date{May 2025}
%TC:endignore

\begin{document}

\maketitle

%TC:ignore
\noindent{\textbf{\Large Uncertainty Quantification of Plasma Filament in the Tokamak \\Scrape-off Layer}}

\begin{figure}[h]
    \centering
    \begin{subfigure}[c]{0.45\linewidth}
        \centering
        \includegraphics[width=0.9\linewidth]{Fig1_plasma_filament.png}
        \normalsize{\caption{Example of a plasma filament along a magnetic field line in the SOL \cite{carralero_experimental_2015}}}
        \label{fig:fig1}
    \end{subfigure}
    \hspace{0.05\linewidth}
    \begin{subfigure}[c]{0.45\linewidth}
        \centering
        \includegraphics[width=0.9\linewidth]{Fig2_blob_movement.png}
        \normalsize{\caption{Density and movement of a blob with average density on the 4 poloidal planes \cite{nespoli_3d_2019}}}
        \label{fig:fig2}
    \end{subfigure}
    \normalsize{\caption{Simulations of a blob in tokamak plasmas}}
\end{figure}

\section*{Plan}
\subsection*{Plasma Filament (Blobs)}
\begin{itemize}
    \item What are plasma filaments
    \begin{itemize}
        \item Found in various magnetic fusion device, occurring in various regimes such as L and H mode \cite{boedo_transport_2003} and ELM mode \cite{ben_ayed_inter-elm_2009}
        \item Important as it affects particle transport and heat flux in SOL, and directly impacts durability and lifetime of plasma facing materials \cite{carralero_experimental_2015, krasheninnikov_recent_2008}
        \item Blob formation \cite{krasheninnikov_recent_2008}:
        \begin{enumerate}
            \item Turbulent processes (or MHD instabilities) causes plasma peel off at outmost layer (curvature and $\nabla$B induced F$\times$B drifts)
            \item Plasma polarization due to gravity drift
            \item Vertical charge separation induces E field ($\perp$ to toroidal B field)
            \item Diamagnetic divergence gives current source in direction $\perp$ to drifts and hence circuit must be closed (through parallel or polarization current) \cite{omotani_effects_2015}
            \begin{itemize}
                \item Polarisation current: leads to inertial evolution and blob pushed by $J\times B$ force
                \item Parallel current: leads to sheath dissipation and current induces potential due to sheath resistivity, blob pushed by $E\times B$ force from potential
            \end{itemize}
        \end{enumerate}
        \item Large amount of 2D and 3D simulations \cite{omotani_effects_2015, easy_three_2014, nespoli_3d_2019, garcia_mechanism_2005, shanahan_fluid_2018} done to model blob structure and dynamic of blob to understand filament transport  
    \end{itemize}
    \item Blob parameters and equations for 2D simulation (specifically blob2d in BOUT++) \cite{omotani_effects_2015}
    \begin{itemize}
        \item Density n
         \begin{itemize}
            \item $L_{\parallel}$ is magnitude, $\phi$ is elctrostatic potential, $\bm{\hat{b}}$ is the unit vector in direction of magnetic field ($\bm{\hat{b}}=\bm{B}/B$), A is amplitude of filament, $\epsilon$ is the ratio of length of axes of the ellipse, $\alpha$ is tilted angle to x direction
        \end{itemize}
        \begin{arrowlist}
            \item $\frac{dn}{dt}=\bm{\hat{b}\cdot g}\times \left(n\nabla\phi-\nabla n\right)+\frac{n\left(1-e^{-\phi}\right)}{L_{\parallel}}+\mu_n\nabla^2_{\perp}n$
            \item $n\left(t=0\right)=n_0\left(1+A\exp{\left(-\frac{{{x'}^2/\epsilon}\,+\,{\epsilon {z'}^2}}{\delta^2}\right)}\right)$
            \item $x'=x\,cos\,\alpha\,+\,z\,sin\,\alpha$
            \item $z'=z\,cos\,\alpha\,-\,x\,sin\,\alpha$
        \end{arrowlist}
        \item Vorticity $\Omega$
            \begin{itemize}
            \item $V_{E\times B}$ is the $E\times B$ velocity
            \end{itemize}
        \begin{arrowlist}
            \item $\frac{d\Omega}{dt}=-\frac{1}{2}\bm{\hat{b}\cdot}\nabla V^2_{E\times B}\times \nabla n-\bm{\hat{b}\cdot g}\times\nabla n+\frac{n\left(1-e^{-\phi}\right)}{L_{\parallel}}+\mu_i\nabla^2_{\perp}\Omega$
            \item $\Omega=\nabla\cdot\left(n\nabla_{\perp}\phi\right)$
        \end{arrowlist}
        \item Filament size $\delta_x$,$\delta_z$
            \begin{itemize}
            \item $\delta$ is the geometric mean of the lengths of the axes
            \end{itemize}
            \begin{arrowlist}
            \item $\delta_x=\delta\sqrt{\frac{\epsilon}{\left(cos^2\alpha\,+\,\epsilon^2sin^2\alpha\right)}}$
            \item $\delta_z=\delta\sqrt{\frac{\epsilon}{\left(\epsilon^2cos^2\alpha\,+\,sin^2\alpha\right)}}$
        \end{arrowlist}
        \item Maximum velocity $V_f$
        \begin{itemize}
            \item Velocity scales with filament size and depends on regime
            \item $\delta_x$ is lengthscale in x-direction. $R_c$ is radius of curvature of magnetic field (major radius), $\beta$ is constant
        \end{itemize}
        \begin{arrowlist}
            \item Polarisation current (narrow filament): $V_f\sim\sqrt{gA\delta_x}$ ($g=\frac{1}{R_c}$)
            \item Parallel current (wide filament): $V_f\sim \frac{L_{\parallel}g}{\delta^2_z}\frac{A}{1+\beta A}$
        \end{arrowlist}
    \end{itemize}
\end{itemize}

\subsection*{Uncertainty Quantification}
    \subsubsection*{Surrogate Model}
    \begin{itemize}
        \item Used to approximate mathematical models and maps inputs to outputs without knowing the relationship between design and output variables \cite{williams_novel_2021}
        \item Gaussian Processes
        \begin{itemize}
            \item What are Gaussian processes \cite{hornsby_gaussian_2024}
            \begin{itemize}    
                \item Machine learning method based on gaussian distributions
                \item For each pairs of variables, compares the correlation between the variables (covariance)
                \item Kernels are functions chosen to assume the distribution, used to generate a covariance matrix and that test points (variables) that are close should yield similar results (small covariance) \cite{duvenaud_automatic_2014}
                \item Training data are then added to covariance matrix for evaluation, and are points kernel functions must pass through and posterior distribution is found (most likely result)
                \item Kernels can be combined to yield better approximations \cite{duvenaud_automatic_2014}
            \end{itemize}
        \end{itemize}
    \end{itemize}
    \subsubsection*{Sensitivity Analysis \cite{wirthl_global_2023} } 
    \begin{itemize}
        \item Uncertainty of model output found using model inputs, including input parameters, boundary and initial conditions etc.
        \item Can help to identify the significance of input parameters
        \item Sobol Method
        \begin{itemize}
            \item Variance based method
            \item Requires large amount of model evaluations
            \item Sobol indices are calculated for each variable, with higher value being more influential
            \item Total-order Sobol indexis then calculated to determine if a parameter is non-influential and can be fixed if determined non-influential
        \end{itemize}
    \end{itemize}
\pagebreak
%TC:endignore

%TC:ignore
\noindent{\textbf{\Large Uncertainty Quantification of Plasma Filament in the Tokamak \\Scrape-off Layer}}
%TC:endignore
\section*{Literature Review}
\subsection*{Overview}
Plasma filaments or blobs are coherent plasma structures that form from perturbations in the Scrape-off Layer (SOL) that transport heat and material across the magnetic field lines \cite{dippolito_convective_2011, hoare_dynamics_2019}. It is useful to perform simulations on the evolution of plasma filaments as it could help to review particle and heat transport of the plasma at the SOL, as well as the impact on the durability and lifetime of plasma facing materials and vessel walls \cite{carralero_experimental_2015, krasheninnikov_recent_2008}. In order to aid simulation, it would be useful to quantify the importance of each of the parameters that govern the properties of the blob.

Uncertainty quantification is a framework that could model the uncertainties of the governing model parameters and the effect it has on the overall system \cite{sudret_surrogate_2017}. Using surrogate models such as Gaussian processes, sensitivity analysis could be performed on the parameters of the governing equations relating to the properties of the plasma filaments.

Through the use of uncertainty quantification, the parameters of the equations that govern plasma filament property could be analysed whilst avoiding being computationally expensive. 

\subsection*{Plasma Filament}
Plasma filaments are a common feature among magnetic fusion devices in different operating modes, appearing in devices such as tokamaks and stellarators and in operating modes including L-mode, H-mode and edge localised modes (ELMs) \cite{ben_ayed_inter-elm_2009, killer_plasma_2020, boedo_transport_2003}. The importance of this feature was due to the discovery of plasma recycling within the main chamber of the reactor, and instead of the fusion plasma flowing to the divertor it was flowing into the chamber walls \cite{krasheninnikov_recent_2008,dippolito_convective_2011}.

The mechanism for blob formation was proposed by Krashennikov \textit{et al.} \cite{krasheninnikov_recent_2008, krasheninnikov_scrape_2001}. At first, the plasma would be separated and formed at the outmost layer of the SOL due to turbulent processes. Plasma polarization of the separated plasma would then occur due to particle drift effects within the vessel, such as the curvature drift and the $\bm{\nabla B}$ drift. The vertical charge separation then induces an $\bm{E}$ field in the perpendicular direction to the toroidal direction, leading to an $\bm{E}\times\bm{B}$ force on the plasma radially outward. An alternative mechanism of the current source was proposed by Omotani \textit{et al.} that a current source was due to the diamagnetic current drift in the plasma instead of the particle drifts. To maintain quasineutrality in the plasma, the circuit would be closed through polarisation current or parallel current configuration.

As reviewed by \cite{dippolito_convective_2011} there are a large amount of  2D and 3D simulations done to understand blob dynamics with different closure schemes for the 2D model that explains the motion and properties of the blob. The most common closure scheme has been the  sheath limited scheme, assuming the filament is in the far SOL and parallel current of the blob would be limited by sheath resistivity. Other examples of closure schemes focus on the effects of enhanced polarization at X-points and magnetic field line bending due to high $\beta$ plasmas\cite{krasheninnikov_recent_2008}. Work has been done to expand the blob models to include other physical phenomena such as drift wave instabilities \cite{angus_effect_2012} and the cause of "Boltzman-like" potential due to blob spinning \cite{angus_effects_2012} to describe 3D effects on blob dynamics that affect the evolution of the 2D blob simulations.

2D and 3D simulations were subsequently compared to review the accuracy of the 2D simulations to the 3D simulations as 3D simulations incorporate effects such as drift wave turbulence and blob spinning. It was concluded in \cite{angus_3d_2012}that the 2D simulation with sheath limited closure is only valid on time scales that were short compared to the time for the drift wave development. More agreeable results between 2D and 3D simulations were found in \cite{easy_three_2014} for sheath dissipation closure and in some cases for vorticity advection closure, but both articles concluded that 2D simulations could not replicate 3D simulation dynamics when drift wave turbulence and blob spinning effects become dominant.

Since this project would be using the BOUT++ framework \cite{dudson_bout_2009} for blob simulation, the governing equations and relevant parameters for blob dynamics were referenced from the work by \cite{omotani_effects_2015}. The mathematical model used in this work considered the filament velocity would be limited by inertial and sheath currents, with the assumption that parallel resistivity would be insignificant for parallel currents to reach the sheath. The variables that describe blob properties include density (n), vorticity ($\Omega$), plasma length scales (in x and z directions, $\delta_x$ and $\delta_z$) and maximum filament velocity ($V_f$), with $V_f$ being the output variable scaling with plasma filament amplitude compared with the background (A), and $\delta_x$ or $\delta_z$, depending on the regime.  

\subsection*{Uncertainty Quantification}
Uncertainty quantification is the process that quantifies the uncertainties in the output by evaluating the uncertainties of the input of a mathematical model, that "deals with assessing, characterising and managing uncertainty in computer models and simulations" \cite{wu_chapter_2024,li_gaussian_2025}. The uncertainty assessed could then be used for functions such as optimisation, sensitivity analysis and parameter estimations for the model. In the context of this project, uncertainty quantification would be used to perform sensitivity analysis on the input parameters for the model under the BOUT++ framework to assess the importance and influence of each parameter on the dynamics of blob simulation.

Sobol method is a common method for performing sensitivity analysis of the model parameter, and has been applied models in other fields including biomechanics \cite{wirthl_global_2023} and pharmacology \cite{zhang}
%TC:ignore
\begin{itemize}
    \item What is Uncertainty Quantification
    \item Surrogate model and gaussian model $\rightarrow$ why gaussian and development of gaussian models
    \item Sobol method
    \item Examples and evolution of applications of surrogate models in high fidelity models
    \item Application of surrogate model in Sobol methods
\end{itemize}
%TC:endignore

\nocite{*}
\printbibliography[title={References}]

%TC:ignore
\detailtexcount{Lit_rev}
%TC:endignore

\end{document}

